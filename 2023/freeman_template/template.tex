%----------------------------------------------------------------------------------------
%	PACKAGES AND OTHER DOCUMENT CONFIGURATIONS
%----------------------------------------------------------------------------------------

\documentclass[
	10pt, % Default font size, can be between 8pt and 12pt
]{FreemanCV}


% Handle images:
\graphicspath{{./images/}}


\columnratio{0.40, 0.60} % Widths of the two columns, specified here as a ratio summing to 1 to correspond to percentages; adjust as needed for your content 

% Headers and footers can be added with the following commands: \lhead{}, \rhead{}, \lfoot{} and \rfoot{}
% Example right footer:
%\rfoot{\textcolor{headings}{\sffamily Last update: \today. Typeset with Xe\LaTeX}}

%----------------------------------------------------------------------------------------

\begin{document}

\begin{paracol}{2} % Begin two-column mode

%----------------------------------------------------------------------------------------
%	YOUR NAME AND CURRICULUM VITAE TITLE
%----------------------------------------------------------------------------------------

\parbox[][0.05\textheight][c]{\linewidth}{ % Box to hold your name and CV title; change the fixed height as needed to match the colored box to the right
	\centering % Horizontally center text
	
	{\sffamily\Huge Josh \textbf{Booth}} % Your name
	
	\medskip % Vertical whitespace
	
	%{\cursivefont\Huge\textcolor{headings}{Big Bad Baller}}
	
	\vfill % Push content to the top of the box
}

%----------------------------------------------------------------------------------------
%	MAJOR RESEARCH PROJECT
%----------------------------------------------------------------------------------------

% \section{Doctral Resoearch}

% {\raggedright\textbf{``Observation of Einstein-Podolsky-Rosen Entanglement on Supraquantum Structures by Induction Through Nonlinear Transuranic Crystal of Extremely Long Wavelength Pulse from Mode-Locked Source Array"}\par}

% \medskip % Vertical whitespace

% My research examined the use of ELW pulses from a mode-locked source array inducted through transuranic crystals to observe entanglement on supraquantum structures. Theoretical advancements included prediction of quantum resonance phenomena including the possibility of resonance cascades. I was motivated to conduct this doctoral research due to my passion for teleportation of matter and I believe I have laid the foundation for further experimental validation and development of practical outcomes.

% \medskip % Extra vertical whitespace before the next section

%----------------------------------------------------------------------------------------
%	WORK EXPERIENCE
%----------------------------------------------------------------------------------------

\section{Work Experience}

% Each job is added with a \jobentry command. Below is an empty one to use as a template:

%\jobentry
%	{} % Duration
%	{} % FT/PT (full time or part time)
%	{} % Employer
%	{} % Job title
%	{} % Description

% All 5 parameters must be supplied but any can be empty if you don't need them

%------------------------------------------------

\jobentry
	{Current, from Jun 2022} % Duration
	{FT} % FT/PT (full time or part time)
	{Microchip Technology} % Employer
	{Applications/Marketing Engineer} % Job title
	{ % Description
		Bridged the gap between technical aspects of 8-bit microcontroller products and their marketing strategies.
		This role involved both developing the initial mass market campaign for multiple products
		and explaining technical concepts in an easily digestible way to the mass market and field sales reps. 
		Internally, I also lead an analytics initiative that lead to better data-driven decisions about our marketing campaign strategies.
	} 

%------------------------------------------------

\jobentry
	{Jun 2017 - Jan 2022} % Duration
	{FT/PT} % FT/PT (full time or part time)
	{United States Naval Research Laboratory} % Employer
	{Electrical Engineer Student Trainee} % Job title
	{ % Description
		Developed machine learning and a data analytic foundation through multiple projects involving
		detecting and classifying objects in satellite imagery to improve field response time,
		recovering noisy RF communications by augmenting a message passing algorithm with a Bidirectional RNN with GRU cells, 
		and identify promoter sequences in DNA using Bidirection RNN with LSTM cells.
	} 

%------------------------------------------------

\jobentry
	{Sep 2016, Jun 2022} % Duration
	{FT/PT} % FT/PT (full time or part time)
	{Booth Oil and Gas LLC.} % Employer
	{Prototype Engineer} % Job title
	{ % Description
		Developed specialized prototypes for novel problems the client faced in both day-to-day operations
		as well as for expanding the business, such as building an human-sized PEEK 3D printer for biofuel refinement.
		I offered technical expertise for project planning, cost estimation, feasibility studies.
		In addition to other project management tasks, I communicated with the client to manage their expectations
		and incorporate or remove features as their needs changed.
	}



	
%----------------------------------------------------------------------------------------
%	EDUCATION
%----------------------------------------------------------------------------------------

\section{Education} 

% Each qualification entry is added with a \qualificationentry command. Below is an empty one to use as a template:

%\qualificationentry
%	{} % Duration
%	{} % Degree
%	{} % Honors, achievements or distinctions (e.g. first class honors)
%	{} % Department
%	{} % Institution

% All 5 parameters must be supplied but any can be empty if you don't need them

%------------------------------------------------

\begin{supertabular}{r l} % Start a table with two columns, the table will ensure everything is aligned

	%------------------------------------------------
	
	\qualificationentry
		{2018-2022} % Duration
		{Computer Engineering} % Degree
		{Summa Cum Laude; 3.98 GPA; Comp Org SI} % Honors, achievements or distinctions (e.g. first class honors)
		{Bachelors of Science} % Department
		{Shippensburg University of Pennsylvania} % Institution
	
	%------------------------------------------------
	
	\qualificationentry
		{2018-2022} % Duration
		{Mathematics} % Degree
		{} % Honors, achievements or distinctions (e.g. first class honors)
		{Minor} % Department
		{Shippensburg University of Pennsylvania} % Institution
	
	%------------------------------------------------
	

	%------------------------------------------------

\end{supertabular}


%----------------------------------------------------------------------------------------
%	REFERENCES
%----------------------------------------------------------------------------------------

% \section{References}

% %\textit{References available on request} % Uncomment if you'd rather not include references and remove the section below

% %------------------------------------------------

% % This section is laid out using a table. A \tableentry command adds lines with the following parameters:

% %\tableentry{Heading}{Content}{spaceafter}
% % All 3 parameters must be supplied but any can be empty if you don't need them
% % A "spaceafter" value in the third parameter will add some vertical space -- this is to be used between headings, leave it empty for no extra space

% %------------------------------------------------

% \begin{supertabular}{r l} % Start a table with two columns, the table will ensure everything is aligned
	
% 	%------------------------------------------------
	
% 	\tableentry{}{\textbf{Dr. Isaac Kleiner}}{spaceafter}
% 	\tableentry{Position}{Professor}{}
% 	\tableentry{Employer}{\href{https://web.mit.edu/physics/}{Department of Physics}}{}
% 	\tableentry{}{\href{https://web.mit.edu}{\textit{Massachusetts Institute of Technology}}}{spaceafter}
% 	\tableentry{Phone}{+1 (617) 253 1000 x5322 (Work)}{}
% 	\tableentry{Mobile}{+1 (232) 842-3583}{}
	
% 	%------------------------------------------------
	
% 	\\ % Additional vertical whitespace between the references
	
% 	%------------------------------------------------
	
% 	\tableentry{}{\textbf{Dr. Eli Vance}}{spaceafter}
% 	\tableentry{Position}{Scientist (HL1)}{}
% 	\tableentry{Employer}{\href{http://www.bmrf.us}{Black Mesa Research Facility}}{spaceafter}
% 	\tableentry{Email}{\href{mailto:e.vance@bmrf.us}{e.vance@bmrf.us}}{}
% 	\tableentry{Phone}{+1 (800) 786-1410 x6235 (Work)}{}
% 	\tableentry{Mobile}{+1 (201) 632-3901}{}
	
% 	%------------------------------------------------
	
% \end{supertabular}

% \medskip % Extra vertical whitespace before the next section

% %----------------------------------------------------------------------------------------

\switchcolumn % Switch to the second (right) column

%----------------------------------------------------------------------------------------
%	COLORED CONTACT DETAILS BOX
%----------------------------------------------------------------------------------------

\parbox[top][0.1\textheight][c]{\linewidth}{ % Box to hold the colored box; change the fixed height as needed to match the box to the left
	\colorbox{shade}{ % Create colored box and specify background color
		\begin{supertabular}{@{\hspace{3pt}} p{0.05\linewidth} | p{0.775\linewidth}} % Start a table with two columns, the table will ensure everything is aligned
			\raisebox{-1pt}{\faPhone} & (717) 494-6466 \\ % Phone number
			\raisebox{-1pt}{\small\faEnvelope} & \href{mailto:boothjmail@gmail.com}{boothjmail@gmail.com} \\ % Email address
			\raisebox{-1pt}{\small\faLink} & \href{https://joshbooth.us}{joshbooth.us} \\ % Website
			\raisebox{-1pt}{\faHome} & Chandler, AZ \\ % Address
			%\raisebox{-1pt}{\faGithub} & \href{https://github.com/username}{https://github.com/username} \\ % GitHub profile
			%\raisebox{-1pt}{\faLinkedinSquare} & \href{https://www.linkedin.com/in/username}{https://www.linkedin.com/in/username} \\ % LinkedIn profile
			% See fontawesome.pdf in the Fonts folder for all icons you can use
		\end{supertabular}
	}
	\vfill % Push content to the top of the box
}


%----------------------------------------------------------------------------------------
%	ENGINEERING PROJECTS
%----------------------------------------------------------------------------------------

\section{Notable Projects}

\subsection{DMX Light Show}

I believe in action long-winded discussions. I listen to everyone's viewpoints and use my judgement to immediately act based on consensus to achieve goals quickly and efficiently.



% Entry 1

\setlength\intextsep{20pt} % how far image is down from section title
\begin{wrapfigure}[4]{rt}{35pt} % # of narrow lines, right top alignment, image L/R adjustment
	\hspace*{-20pt} % how close horizontal text can be
	\vspace*{-20pt}
    \includegraphics[scale=0.6]{security_system} %Z how large image is
\end{wrapfigure}

\vspace*{-15pt} % move image's anchor box up closer to title
\leavevmode\subsection{\href{https://github.com/jfcbooth/security_system}{AI-Driven Security System \scriptsize\faLink}}


A wireless, solar-powered securty system that provides real-time alerts through AI-identifcation of humans, vehicles, and large animals.

% Entry 2

% %\setlength\intextsep{20pt} % how far image is down from section title
% \begin{wrapfigure}{L}{0.05\textwidth}
% 	%\hspace*{-20pt} % how close horizontal text can be
% 	%\vspace*{-80pt}
%     \includegraphics[scale=0.04]{printer} %Z how large image is
% \end{wrapfigure}

% \vspace*{-5pt} % move image's anchor box up closer to title
% \leavevmode \subsection{3D PEEK Printer}

\setlength\intextsep{17pt} % how far image is down from section title
\begin{wrapfigure}[9]{l}{28pt}
	%\hspace*{-20pt} % how close horizontal text can be
	%\vspace*{-80pt}
    \includegraphics[scale=0.175]{printer2} %Z how large image is
\end{wrapfigure}

%\vspace*{-5pt} % move image's anchor box up closer to title
\leavevmode \subsection{3D PEEK Printer}

% \noindent\raisebox{30pt}{\includegraphics[scale=0.04]{printer}}\hfill %Z how large image is
% \subsection{3D PEEK Printer}

I have been interested in theoretical physics such as quantum mechanics and relativity from an early age. My education and research have cemented this interest into a passion. I greatly enjoy carrying out fundamental physics research with potential practical applications.

\subsection{The Cold Plate}

The cold Plate


%----------------------------------------------------------------------------------------
%	AWARDS
%----------------------------------------------------------------------------------------

\section{Awards}

% This section is laid out using a table. A \tableentry command adds lines with the following parameters:

%\tableentry{Heading}{Content}{spaceafter}
% All 3 parameters must be supplied but any can be empty if you don't need them
% A "spaceafter" value in the third parameter will add some vertical space -- this is to be used between headings, leave it empty for no extra space

%------------------------------------------------

\begin{supertabular}{r l} % Start a table with two columns, the table will ensure everything is aligned
	
	%------------------------------------------------
	
	\tableentry{1985}{\textbf{Faculty of Science Masters Scholarship}}{}
	\tableentry{}{\textit{Massachusetts Institute of Technology}}{spaceafter}
	
	%------------------------------------------------
	
	\tableentry{1983}{\textbf{Top Achiever Award -- Physics}}{}
	\tableentry{}{\textit{The University of Washington}}{spaceafter}
	
	%------------------------------------------------
	
\end{supertabular}

%----------------------------------------------------------------------------------------
%	COMPUTER SKILLS
%----------------------------------------------------------------------------------------

% \section{Misc. Skills} 

% % This section is laid out using a table. A \tableentry command adds lines with the following parameters:

% %\tableentry{Heading}{Content}{spaceafter}
% % All 3 parameters must be supplied but any can be empty if you don't need them
% % A "spaceafter" value in the third parameter will add some vertical space -- this is to be used between headings, leave it empty for no extra space

% %------------------------------------------------

% \begin{supertabular}{r l} % Start a table with two columns, the table will ensure everything is aligned
	
% 	%------------------------------------------------
	
% 	\tableentry{Beginner}{Java, MS DOS}{spaceafter}
	
% 	%------------------------------------------------
	
% 	\tableentry{Intermediate}{Javascript, Python, HTML, CSS,}{}
% 	\tableentry{}{Microsoft Windows}{}
% 	\tableentry{}{Computer Hardware \& Support}{spaceafter}
	
% 	%------------------------------------------------
	
% 	\tableentry{Expert}{Perl, Unix, \LaTeX}{spaceafter}
	
% 	%------------------------------------------------
	
% \end{supertabular}

%----------------------------------------------------------------------------------------
%	COMMUNICATION SKILLS
%----------------------------------------------------------------------------------------

% \section{Communication Skills}

% % This section is laid out using a table. A \tableentry command adds lines with the following parameters:

% %\tableentry{Heading}{Content}{spaceafter}
% % All 3 parameters must be supplied but any can be empty if you don't need them
% % A "spaceafter" value in the third parameter will add some vertical space -- this is to be used between headings, leave it empty for no extra space

% %------------------------------------------------

% \begin{supertabular}{r l} % Start a table with two columns, the table will ensure everything is aligned
	
% 	%------------------------------------------------
	
% 	\tableentry{Conferences}{Oral Presentation at the Annual MIT}{}
% 	\tableentry{}{Theoretical Physics Conference -- 1987}{spaceafter}
	
% 	%------------------------------------------------
	
% 	\tableentry{Posters}{Poster at the Meeting of the American}{}
% 	\tableentry{}{Physical Society -- 1985}{spaceafter}
	
% 	%------------------------------------------------
	
% \end{supertabular}


%----------------------------------------------------------------------------------------
%	PUBLICATIONS
%----------------------------------------------------------------------------------------

\section{Publications}

%------------------------------------------------

% \href{https://ww1.microchip.com/downloads/aemDocuments/documents/MCU08/ApplicationNotes/ApplicationNotes/AN4889-Using-CIPs-To-Implement-Peltier-Plate-DS00004889.pdf}{Using Core Independent Peripherals (CIPs) to Implement a
% 	Peltier Cooled Metal Plate}

% \medskip % Vertical whitespace

% Jacobsen, F. M., Gee, N., \textbf{Freeman, G. R.} (1986). Electron mobility in liquid krypton as function of density, temperature, and electric field strength. \textit{Physical Review A}, \textit{34}(3): 2329-2335.

% \medskip % Vertical whitespace

%------------------------------------------------

% As an alternative to a long-form publication list, you can create a shorter summary using only DOI values and years.

% Example \doipublication{} command to add another publication:

%\doipublication{Year}{DOI}{firstauthor}{spaceafter}

% All four parameters are required (can be empty though)
% A value of "firstauthor" in the third parameter will output the DOI in bold
% A "spaceafter" value in the fourth parameter will add some vertical space -- this is to be used between years

%------------------------------------------------

\begin{supertabular}{p{0.05\linewidth} p{0.95\linewidth}} % Start a table with two columns, the table will ensure everything is aligned
	
	%------------------------------------------------
	2023 & \href{
	https://ww1.microchip.com/downloads/aemDocuments/documents/MCU08/ApplicationNotes/ApplicationNotes/AN4889-Using-CIPs-To-Implement-Peltier-Plate-DS00004889.pdf
	}{
	Using Core Independent Peripherals (CIPs) to Implement a Peltier Cooled Metal Plate
	\scriptsize\faLink} \\
	%------------------------------------------------
	2018 & \href{
	http://joshbooth.us/wp-content/uploads/2023/08/Machine-Learning-in-Radio-Frequency-Communications.pdf
	}{
	Machine Learning in Radio Frequency Communications
	\scriptsize\faLink} \\
	%------------------------------------------------
	2017 & \href{
	http://joshbooth.us/wp-content/uploads/2023/08/poster_SBME_promoter_predictions.pdf
	}{
	Prediction of Bacterial Promoter Sequences using Machine Learning
	\scriptsize\faLink} \\
	%------------------------------------------------
	
\end{supertabular}

\medskip % Extra whitespace before the next section

%----------------------------------------------------------------------------------------

\end{paracol} % End two-column mode

%----------------------------------------------------------------------------------------

\end{document}
